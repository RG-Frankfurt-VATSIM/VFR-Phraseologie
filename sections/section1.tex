%!TEX root = ../VFR_Phraseologie.tex

\subsection{Zahlen}
\begin{table}[H]
	\begin{tabularx}{\textwidth}{XX}
		\textbf{English} & \textbf{Deutsch} \\
		zero             & Null             \\
		one              & Eins             \\
		two              & Zw\textbf{o}     \\
		t\textbf{r}ee    & Drei             \\
		four             & Vier             \\
		five             & Fünf             \\
		six              & Sechs            \\
		seven            & Sieben           \\
		eight            & Acht             \\
		nine\textbf{r}   & Neun             \\
		hundred          & Hundert          \\
		tousand          & Tausend         
	\end{tabularx}%
\end{table}

\subsection{NATO-Alphabet}
\begin{table}[H]
	\begin{tabularx}{\textwidth}{XX}
		A & Alfa                      \\
		B & Bravo                     \\
		C & Charlie                   \\
		D & Delta                     \\
		E & Echo (ausgesprochen Ekko) \\
		F & Foxtrot                   \\
		G & Golf                      \\
		H & Hotel                     \\
		I & India                     \\
		J & Juliet                    \\
		K & Kilo                      \\
		L & Lima                      \\
		M & Mike                      \\
		N & November                  \\
		O & Oscar                     \\
		P & Papa                      \\
		Q & Quebec                    \\
		R & Romeo                     \\
		S & Sierra                    \\
		T & Tango                     \\
		U & Uniform                   \\
		V & Viktor                    \\
		W & Whiskey                   \\
		X & X-Ray                     \\
		Y & Yankee                    \\
		Z & Zulu                     
	\end{tabularx}
\end{table}

\subsection{Einheiten}
\begin{table}[H]
	\begin{tabularx}{\textwidth}{XX}
		\textbf{English} 	& \textbf{Deutsch}      \\
		feet        		& Fuß        \\
		flightlevel 		& Flugfläche \\
		degrees     		& Grad       \\
		knot        		& Knoten
	\end{tabularx}
\end{table}
\subsection{Aerodrome Information}
\begin{table}[H]
	\begin{tabularx}{\textwidth}{XX}
		\textbf{English} 				& \textbf{Deutsch}     								   \\
		active runways               & Aktive Pisten               \\
		QNH (englisch ausgesprochen) & QNH (deutsch ausgesprochen)                        
	\end{tabularx}
\end{table}
\subsection{Wind}
\begin{table}[H]
	\begin{tabularx}{\textwidth}{XX}
		\textbf{English} 				& \textbf{Deutsch}     								   \\
		wind calm                  & Wind still               \\
		wind XXX degrees, XX knots & Wind XXX Grad, XX Knoten
	\end{tabularx}
\end{table}

\subsection{Taxiing}
\begin{table}[H]
	\begin{tabularx}{\textwidth}{XX}
		\textbf{English} 				& \textbf{Deutsch}     								   \\
		runway                        	& Piste                                                \\
		taxiway                       	& Rollweg                                              \\
		holding point                 	& Rollhalt                                             \\
		line up runway 25C (and wait) 	& Rollen Sie zum Abflugpunkt Piste 25C (dort halten)   \\
		General Aviation Parking     	& Parkplatz der allgemeinen Luftfahrt (oder kurz: GAT)
	\end{tabularx}
\end{table}
\textbf{Examples:}
\begin{table}[H]
	\begin{tabularx}{\textwidth}{XX}
		\textbf{English} 															& \textbf{Deutsch}      															   \\
		Taxi to holding point runway 25C via S23, R11, M19, M24, hold short of 25L. & Rollen Sie zum Rollhalt Piste 25C über S23, R11, M19, M24, halten Sie vor Piste 25L. \\
		Continue taxi, cross runway 25L.                                            & Setzen Sie Rollen fort, überqueren Sie Piste 25L.
	\end{tabularx}
\end{table}

\subsection{Freigaben}
\begin{table}[H]
	\begin{tabularx}{\textwidth}{XX}
		\textbf{English} 					   & \textbf{Deutsch}     				 \\
		cleared for takeoff                    & Start frei                          \\ 
		cleared to land                        & Landung frei                        \\
		cleared low approach                   & frei zum Tiefanflug                 \\
		cleared touch and go                   & frei zum Aufsetzen und Durchstarten \\
		Go around!                             & Starten Sie durch!                  \\
		Are you ready for immediate departure? & Sind sie bereit zum Sofortabflug?   \\
		cleared for immediate takeoff          & frei zum Sofortstart                \\
		Do not overfly runway 18               & Überfliegen Sie nicht Piste 18     
	\end{tabularx}
\end{table}


\subsection{VFR-Traffic circuit}
\begin{table}[H]
	\begin{tabularx}{\textwidth}{XX}
		\textbf{English} 	  & \textbf{Deutsch} \\
		traffic circuit       & Platzrunde       \\
		right traffic circuit & Rechtsplatzrunde \\
		upwind                & Abflug           \\
		crosswind             & Querabflug       \\
		downwind              & Gegenanflug      \\
		base                  & Queranflug       \\
		final                 & Endanflug       
	\end{tabularx}
\end{table}
\textbf{Examples:}
\begin{table}[H]
	\begin{tabularx}{\textwidth}{XX}
		\textbf{English} 	  				   & \textbf{Deutsch} 								\\
		Join right traffic circuit runway 07R. & Fliegen Sie in die Rechtsplatzrunde Piste 07R. \\
		Join direct final runway 07R.          & Fliegen Sie direkt in den Endanflug Piste 07R.
	\end{tabularx}
\end{table}


\subsection{Waiting}
\begin{table}[H]
	\begin{tabularx}{\textwidth}{XX}
		\textbf{English} 					& \textbf{Deutsch}     								   	\\
		make a left/right three-sixty     	& Machen Sie einen Vollkreis links/rechts               \\
		orbit left/right   					& Kreisen Sie links/rechts           					\\
		hold over XX       					& Halten Sie über XX                 					\\
		abeam threshold XX 					& auf Höhe der (Pisten-) Schwelle XX 					\\
		extend downwind, standby for base 	& Verlängern Sie Gegenanflug, warten Sie auf Queranflug
	\end{tabularx}
\end{table}

\subsection{Stationen}
\begin{table}[H]
	\begin{tabularx}{\textwidth}{XX}
		\textbf{English} 					& \textbf{Deutsch} 				   \\
		delivery							&-   							   \\
		apron								&Vorfeld   						   \\
		ground								&Rollkontrolle   				   \\
		tower								&Turm     						   \\
		radar								&Radar						   	   \\
		information							&Information					   
	\end{tabularx}
\end{table}

\subsection{Radio}
\begin{table}[H]
	\begin{tabularx}{\textwidth}{XX}
		\textbf{English} 					& \textbf{Deutsch}                 \\
		contact XX (on) XXX decimal XX		& rufen Sie XX (auf) XXX Komma XX  \\
		say again                     		& wiederholen Sie                  \\
		correction                    		& Berichtigung                     \\
		disregard (last transmission)		& ignorieren Sie (letzte Meldung)  \\
		Maintain radio discipline!   		& Halten Sie Funkdisziplin!        \\
		read you {[}1, 2, 3, 4, 5{]} 		& verstehe Sie {[}1, 2, 3, 4, 5{]} \\
		break-break                   		& Trennung                        
	\end{tabularx}
\end{table}